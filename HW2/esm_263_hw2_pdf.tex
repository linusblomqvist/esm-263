% Options for packages loaded elsewhere
\PassOptionsToPackage{unicode}{hyperref}
\PassOptionsToPackage{hyphens}{url}
%
\documentclass[
]{article}
\usepackage{lmodern}
\usepackage{amssymb,amsmath}
\usepackage{ifxetex,ifluatex}
\ifnum 0\ifxetex 1\fi\ifluatex 1\fi=0 % if pdftex
  \usepackage[T1]{fontenc}
  \usepackage[utf8]{inputenc}
  \usepackage{textcomp} % provide euro and other symbols
\else % if luatex or xetex
  \usepackage{unicode-math}
  \defaultfontfeatures{Scale=MatchLowercase}
  \defaultfontfeatures[\rmfamily]{Ligatures=TeX,Scale=1}
\fi
% Use upquote if available, for straight quotes in verbatim environments
\IfFileExists{upquote.sty}{\usepackage{upquote}}{}
\IfFileExists{microtype.sty}{% use microtype if available
  \usepackage[]{microtype}
  \UseMicrotypeSet[protrusion]{basicmath} % disable protrusion for tt fonts
}{}
\makeatletter
\@ifundefined{KOMAClassName}{% if non-KOMA class
  \IfFileExists{parskip.sty}{%
    \usepackage{parskip}
  }{% else
    \setlength{\parindent}{0pt}
    \setlength{\parskip}{6pt plus 2pt minus 1pt}}
}{% if KOMA class
  \KOMAoptions{parskip=half}}
\makeatother
\usepackage{xcolor}
\IfFileExists{xurl.sty}{\usepackage{xurl}}{} % add URL line breaks if available
\IfFileExists{bookmark.sty}{\usepackage{bookmark}}{\usepackage{hyperref}}
\hypersetup{
  pdftitle={ESM 263 HW2},
  pdfauthor={Linus Blomqvist},
  hidelinks,
  pdfcreator={LaTeX via pandoc}}
\urlstyle{same} % disable monospaced font for URLs
\usepackage[margin=1in]{geometry}
\usepackage{color}
\usepackage{fancyvrb}
\newcommand{\VerbBar}{|}
\newcommand{\VERB}{\Verb[commandchars=\\\{\}]}
\DefineVerbatimEnvironment{Highlighting}{Verbatim}{commandchars=\\\{\}}
% Add ',fontsize=\small' for more characters per line
\usepackage{framed}
\definecolor{shadecolor}{RGB}{248,248,248}
\newenvironment{Shaded}{\begin{snugshade}}{\end{snugshade}}
\newcommand{\AlertTok}[1]{\textcolor[rgb]{0.94,0.16,0.16}{#1}}
\newcommand{\AnnotationTok}[1]{\textcolor[rgb]{0.56,0.35,0.01}{\textbf{\textit{#1}}}}
\newcommand{\AttributeTok}[1]{\textcolor[rgb]{0.77,0.63,0.00}{#1}}
\newcommand{\BaseNTok}[1]{\textcolor[rgb]{0.00,0.00,0.81}{#1}}
\newcommand{\BuiltInTok}[1]{#1}
\newcommand{\CharTok}[1]{\textcolor[rgb]{0.31,0.60,0.02}{#1}}
\newcommand{\CommentTok}[1]{\textcolor[rgb]{0.56,0.35,0.01}{\textit{#1}}}
\newcommand{\CommentVarTok}[1]{\textcolor[rgb]{0.56,0.35,0.01}{\textbf{\textit{#1}}}}
\newcommand{\ConstantTok}[1]{\textcolor[rgb]{0.00,0.00,0.00}{#1}}
\newcommand{\ControlFlowTok}[1]{\textcolor[rgb]{0.13,0.29,0.53}{\textbf{#1}}}
\newcommand{\DataTypeTok}[1]{\textcolor[rgb]{0.13,0.29,0.53}{#1}}
\newcommand{\DecValTok}[1]{\textcolor[rgb]{0.00,0.00,0.81}{#1}}
\newcommand{\DocumentationTok}[1]{\textcolor[rgb]{0.56,0.35,0.01}{\textbf{\textit{#1}}}}
\newcommand{\ErrorTok}[1]{\textcolor[rgb]{0.64,0.00,0.00}{\textbf{#1}}}
\newcommand{\ExtensionTok}[1]{#1}
\newcommand{\FloatTok}[1]{\textcolor[rgb]{0.00,0.00,0.81}{#1}}
\newcommand{\FunctionTok}[1]{\textcolor[rgb]{0.00,0.00,0.00}{#1}}
\newcommand{\ImportTok}[1]{#1}
\newcommand{\InformationTok}[1]{\textcolor[rgb]{0.56,0.35,0.01}{\textbf{\textit{#1}}}}
\newcommand{\KeywordTok}[1]{\textcolor[rgb]{0.13,0.29,0.53}{\textbf{#1}}}
\newcommand{\NormalTok}[1]{#1}
\newcommand{\OperatorTok}[1]{\textcolor[rgb]{0.81,0.36,0.00}{\textbf{#1}}}
\newcommand{\OtherTok}[1]{\textcolor[rgb]{0.56,0.35,0.01}{#1}}
\newcommand{\PreprocessorTok}[1]{\textcolor[rgb]{0.56,0.35,0.01}{\textit{#1}}}
\newcommand{\RegionMarkerTok}[1]{#1}
\newcommand{\SpecialCharTok}[1]{\textcolor[rgb]{0.00,0.00,0.00}{#1}}
\newcommand{\SpecialStringTok}[1]{\textcolor[rgb]{0.31,0.60,0.02}{#1}}
\newcommand{\StringTok}[1]{\textcolor[rgb]{0.31,0.60,0.02}{#1}}
\newcommand{\VariableTok}[1]{\textcolor[rgb]{0.00,0.00,0.00}{#1}}
\newcommand{\VerbatimStringTok}[1]{\textcolor[rgb]{0.31,0.60,0.02}{#1}}
\newcommand{\WarningTok}[1]{\textcolor[rgb]{0.56,0.35,0.01}{\textbf{\textit{#1}}}}
\usepackage{graphicx,grffile}
\makeatletter
\def\maxwidth{\ifdim\Gin@nat@width>\linewidth\linewidth\else\Gin@nat@width\fi}
\def\maxheight{\ifdim\Gin@nat@height>\textheight\textheight\else\Gin@nat@height\fi}
\makeatother
% Scale images if necessary, so that they will not overflow the page
% margins by default, and it is still possible to overwrite the defaults
% using explicit options in \includegraphics[width, height, ...]{}
\setkeys{Gin}{width=\maxwidth,height=\maxheight,keepaspectratio}
% Set default figure placement to htbp
\makeatletter
\def\fps@figure{htbp}
\makeatother
\setlength{\emergencystretch}{3em} % prevent overfull lines
\providecommand{\tightlist}{%
  \setlength{\itemsep}{0pt}\setlength{\parskip}{0pt}}
\setcounter{secnumdepth}{-\maxdimen} % remove section numbering
\usepackage{setspace}\doublespacing
\usepackage{booktabs}
\usepackage{longtable}
\usepackage{array}
\usepackage{multirow}
\usepackage{wrapfig}
\usepackage{float}
\usepackage{colortbl}
\usepackage{pdflscape}
\usepackage{tabu}
\usepackage{threeparttable}
\usepackage{threeparttablex}
\usepackage[normalem]{ulem}
\usepackage{makecell}
\usepackage{xcolor}

\title{ESM 263 HW2}
\author{Linus Blomqvist}
\date{2021-01-26}

\begin{document}
\maketitle

The .Rmd files with the code, and the html file with interactive maps in
the output can both be found in my GitHub repo
\href{https://github.com/linusblomqvist/esm-263/tree/main/HW2}{here}.

\hypertarget{loading-and-exploring-data}{%
\subsubsection{Loading and exploring
data}\label{loading-and-exploring-data}}

Check what layers are in the basemap:

\begin{Shaded}
\begin{Highlighting}[]
\KeywordTok{st_layers}\NormalTok{(}\StringTok{"HW2/data_hw2/basemap.gpkg"}\NormalTok{)}\OperatorTok{$}\NormalTok{name}
\end{Highlighting}
\end{Shaded}

\begin{verbatim}
## [1] "California" "Cities"     "County"     "ROI"        "Streets"
\end{verbatim}

Read them in:

\begin{Shaded}
\begin{Highlighting}[]
\NormalTok{california <-}\StringTok{ }\KeywordTok{read_sf}\NormalTok{(}\StringTok{"HW2/data_hw2/basemap.gpkg"}\NormalTok{, }\DataTypeTok{layer =} \StringTok{"California"}\NormalTok{)}
\NormalTok{cities <-}\StringTok{ }\KeywordTok{read_sf}\NormalTok{(}\StringTok{"HW2/data_hw2/basemap.gpkg"}\NormalTok{, }\DataTypeTok{layer =} \StringTok{"Cities"}\NormalTok{)}
\NormalTok{county <-}\StringTok{ }\KeywordTok{read_sf}\NormalTok{(}\StringTok{"HW2/data_hw2/basemap.gpkg"}\NormalTok{, }\DataTypeTok{layer =} \StringTok{"County"}\NormalTok{)}
\NormalTok{ROI <-}\StringTok{ }\KeywordTok{read_sf}\NormalTok{(}\StringTok{"HW2/data_hw2/basemap.gpkg"}\NormalTok{, }\DataTypeTok{layer =} \StringTok{"ROI"}\NormalTok{)}
\NormalTok{streets <-}\StringTok{ }\KeywordTok{read_sf}\NormalTok{(}\StringTok{"HW2/data_hw2/basemap.gpkg"}\NormalTok{, }\DataTypeTok{layer =} \StringTok{"streets"}\NormalTok{)}
\end{Highlighting}
\end{Shaded}

For this and any other layers we can do a little exploring. For example,
we can check the variable names:

\begin{Shaded}
\begin{Highlighting}[]
\KeywordTok{names}\NormalTok{(california)}
\end{Highlighting}
\end{Shaded}

\begin{verbatim}
##  [1] "OBJECTID"     "NAME_PCASE"   "NAME_UCASE"   "FMNAME_PC"    "FMNAME_UC"   
##  [6] "ABBREV"       "NUM"          "ABCODE"       "FIPS"         "ANSI"        
## [11] "ISLAND"       "Shape_Leng"   "Shape_Length" "Shape_Area"   "geom"
\end{verbatim}

Note that in an \texttt{sf} object, there's always a column at the end
called \texttt{geom}; this contains all the spatial information.

We can also do a quick plot to see what we have in the
\texttt{california} shape. Looks like counties.

\begin{Shaded}
\begin{Highlighting}[]
\KeywordTok{tm_shape}\NormalTok{(}\KeywordTok{st_geometry}\NormalTok{(california)) }\OperatorTok{+}
\StringTok{  }\KeywordTok{tm_polygons}\NormalTok{()}
\end{Highlighting}
\end{Shaded}

\includegraphics{esm_263_hw2_pdf_files/figure-latex/map_CA-1.pdf}

I'm also interested to see the ROI (region of interest). In the html
version of this file, you'll be able to see this against a base layer
and zoom around etc, as the map view mode is interactive, but in the pdf
version, it's a static map. We can see that we're only interested in
what looks like the downtown area including the harbor.

For the static map, I use the \texttt{read\_osm} function, where osm
stands for Open Street Map, to create a base layer. This requires a
bounding box which is represented by the Santa Barbara city feature.
This is a multipolygon with two parts, the actual city, and the airport,
so I turn it into a polygon with two separate features, and select the
second one, Santa Barbara proper.

\begin{Shaded}
\begin{Highlighting}[]
\CommentTok{# Create feature for bounding box and read Open Street Map layer}
\NormalTok{sb <-}\StringTok{ }\KeywordTok{st_geometry}\NormalTok{(cities[cities}\OperatorTok{$}\NormalTok{CITY }\OperatorTok{==}\StringTok{ "Santa Barbara"}\NormalTok{,])}
\NormalTok{sb_polygon <-}\StringTok{ }\KeywordTok{st_cast}\NormalTok{(sb, }\DataTypeTok{to =} \StringTok{"POLYGON"}\NormalTok{)}
\NormalTok{osm <-}\StringTok{ }\KeywordTok{read_osm}\NormalTok{(}\KeywordTok{st_buffer}\NormalTok{(sb_polygon[}\DecValTok{2}\NormalTok{], }\DecValTok{300}\NormalTok{), }\DataTypeTok{type =} \StringTok{"osm"}\NormalTok{)  }\CommentTok{# use buffer to}
\CommentTok{# have a bit of margin around the map}

\CommentTok{# Map}
\KeywordTok{tm_shape}\NormalTok{(osm) }\OperatorTok{+}
\StringTok{  }\KeywordTok{tm_rgb}\NormalTok{(}\DataTypeTok{alpha =} \FloatTok{0.7}\NormalTok{) }\OperatorTok{+}
\StringTok{  }\KeywordTok{tm_shape}\NormalTok{(sb_polygon[}\DecValTok{2}\NormalTok{]) }\OperatorTok{+}
\StringTok{  }\KeywordTok{tm_borders}\NormalTok{(}\DataTypeTok{col =} \StringTok{"black"}\NormalTok{, }\DataTypeTok{lwd =} \DecValTok{2}\NormalTok{) }\OperatorTok{+}
\StringTok{  }\KeywordTok{tm_shape}\NormalTok{(ROI) }\OperatorTok{+}
\StringTok{  }\KeywordTok{tm_borders}\NormalTok{(}\DataTypeTok{col =} \StringTok{"red"}\NormalTok{, }\DataTypeTok{lwd =} \DecValTok{3}\NormalTok{) }\OperatorTok{+}
\StringTok{  }\KeywordTok{tm_add_legend}\NormalTok{(}\DataTypeTok{type =} \StringTok{"line"}\NormalTok{, }\DataTypeTok{lwd =} \DecValTok{3}\NormalTok{, }\DataTypeTok{col =} \StringTok{"red"}\NormalTok{, }\DataTypeTok{labels =} \StringTok{"ROI"}\NormalTok{, }\DataTypeTok{alpha =} \DecValTok{1}\NormalTok{) }\OperatorTok{+}
\StringTok{  }\KeywordTok{tm_add_legend}\NormalTok{(}\DataTypeTok{type =} \StringTok{"line"}\NormalTok{, }\DataTypeTok{lwd =} \DecValTok{2}\NormalTok{, }\DataTypeTok{col =} \StringTok{"black"}\NormalTok{, }\DataTypeTok{labels =} \StringTok{"Santa Barbara"}\NormalTok{, }\DataTypeTok{alpha =} \DecValTok{1}\NormalTok{) }\OperatorTok{+}
\StringTok{  }\KeywordTok{tm_layout}\NormalTok{(}\DataTypeTok{legend.position =} \KeywordTok{c}\NormalTok{(}\StringTok{"left"}\NormalTok{, }\StringTok{"bottom"}\NormalTok{))}
\end{Highlighting}
\end{Shaded}

\includegraphics{esm_263_hw2_pdf_files/figure-latex/unnamed-chunk-3-1.pdf}

There's only one layer, ``parcels'', in the parcels file so we'll read
that in.

\begin{Shaded}
\begin{Highlighting}[]
\NormalTok{parcels <-}\StringTok{ }\KeywordTok{read_sf}\NormalTok{(}\StringTok{"HW2/data_hw2/parcels.gpkg"}\NormalTok{)}
\end{Highlighting}
\end{Shaded}

The variable we're interested in is \texttt{NET\_AV}, so we can check
what that looks like (on a log 10 scale to make it easier to read).

\begin{Shaded}
\begin{Highlighting}[]
\KeywordTok{ggplot}\NormalTok{(parcels) }\OperatorTok{+}
\StringTok{  }\KeywordTok{geom_histogram}\NormalTok{(}\KeywordTok{aes}\NormalTok{(}\DataTypeTok{x =}\NormalTok{ NET_AV)) }\OperatorTok{+}
\StringTok{  }\KeywordTok{scale_x_log10}\NormalTok{() }\OperatorTok{+}
\StringTok{  }\KeywordTok{xlab}\NormalTok{(}\StringTok{"Net assessed value"}\NormalTok{)}
\end{Highlighting}
\end{Shaded}

\includegraphics{esm_263_hw2_pdf_files/figure-latex/hist-1.pdf}

Seems like most parcels are valued at just under a million dollars, but
some are worth tens of millions of dollars.

For the inundation scenarios, I combine all the layers into a single
\texttt{sf} object.

\begin{Shaded}
\begin{Highlighting}[]
\CommentTok{# Get layer names}
\NormalTok{inund_layers <-}\StringTok{ }\KeywordTok{st_layers}\NormalTok{(}\StringTok{"HW2/data_hw2/inundation_scenarios.gpkg"}\NormalTok{)}\OperatorTok{$}\NormalTok{name}

\CommentTok{# Start with one layer and then row bind the others onto it with a loop}
\NormalTok{scenarios <-}\StringTok{ }\KeywordTok{st_read}\NormalTok{(}\StringTok{"HW2/data_hw2/inundation_scenarios.gpkg"}\NormalTok{, }\DataTypeTok{layer =}\NormalTok{ inund_layers[}\DecValTok{1}\NormalTok{], }\DataTypeTok{quiet =} \OtherTok{TRUE}\NormalTok{)}
\ControlFlowTok{for}\NormalTok{(i }\ControlFlowTok{in} \DecValTok{2}\OperatorTok{:}\KeywordTok{length}\NormalTok{(inund_layers)) \{}
\NormalTok{  scenarios <-}\StringTok{ }\KeywordTok{rbind}\NormalTok{(scenarios, }\KeywordTok{read_sf}\NormalTok{(}\StringTok{"HW2/data_hw2/inundation_scenarios.gpkg"}\NormalTok{, }\DataTypeTok{layer =}\NormalTok{ inund_layers[i]))}
\NormalTok{\}}
\end{Highlighting}
\end{Shaded}

Let's look at one of these scenarios:

\begin{Shaded}
\begin{Highlighting}[]
\CommentTok{# New bounding box for the base layer}
\NormalTok{osm_ROI <-}\StringTok{ }\KeywordTok{read_osm}\NormalTok{(}\KeywordTok{st_buffer}\NormalTok{(ROI, }\DecValTok{500}\NormalTok{), }\DataTypeTok{type =} \StringTok{"osm"}\NormalTok{)}

\CommentTok{# Map}
\KeywordTok{tm_shape}\NormalTok{(osm_ROI) }\OperatorTok{+}
\StringTok{  }\KeywordTok{tm_rgb}\NormalTok{(}\DataTypeTok{alpha =} \FloatTok{0.7}\NormalTok{) }\OperatorTok{+}
\StringTok{  }\KeywordTok{tm_shape}\NormalTok{(}\KeywordTok{st_geometry}\NormalTok{(}\KeywordTok{filter}\NormalTok{(scenarios, GRIDCODE }\OperatorTok{==}\StringTok{ }\DecValTok{10}\NormalTok{))) }\OperatorTok{+}
\StringTok{  }\KeywordTok{tm_fill}\NormalTok{(}\DataTypeTok{col =} \StringTok{"lightblue"}\NormalTok{, }\DataTypeTok{alpha =} \FloatTok{0.8}\NormalTok{)}
\end{Highlighting}
\end{Shaded}

\includegraphics{esm_263_hw2_pdf_files/figure-latex/map_scenarios-1.pdf}

Seems like this represents current land area that would be inundated
under the scenario in question.

\hypertarget{spatial-join}{%
\subsubsection{Spatial join}\label{spatial-join}}

What we want here is the total value of all parcels that fall within the
inundated area for each scenario.

\begin{Shaded}
\begin{Highlighting}[]
\CommentTok{# Calculate the parcel areas}
\NormalTok{parcels}\OperatorTok{$}\NormalTok{area <-}\StringTok{ }\KeywordTok{st_area}\NormalTok{(parcels) }\CommentTok{# calculate area}
\KeywordTok{units}\NormalTok{(parcels}\OperatorTok{$}\NormalTok{area) <-}\StringTok{ }\KeywordTok{with}\NormalTok{(ud_units, ha) }\CommentTok{# convert from m^2 to ha}
\NormalTok{parcels}\OperatorTok{$}\NormalTok{area <-}\StringTok{ }\KeywordTok{drop_units}\NormalTok{(parcels}\OperatorTok{$}\NormalTok{area) }\CommentTok{# roundabout way of doing this,}
\CommentTok{# but it is less prone to human error to use the units package for conversions}

\CommentTok{# Do the join and summarize for each of the three variables}
\NormalTok{scenarios <-}\StringTok{ }\NormalTok{scenarios }\OperatorTok
\StringTok{  }\KeywordTok{st_join}\NormalTok{(parcels, }\DataTypeTok{join =}\NormalTok{ st_intersects) }\OperatorTok
\StringTok{  }\KeywordTok{group_by}\NormalTok{(GRIDCODE) }\OperatorTok\StringTok{ }\CommentTok{# this is the ID of each scenario}
\StringTok{  }\KeywordTok{summarize}\NormalTok{(}\DataTypeTok{parcel_count =} \KeywordTok{n}\NormalTok{(), }\DataTypeTok{net_value =} \KeywordTok{round}\NormalTok{(}\KeywordTok{sum}\NormalTok{(NET_AV)}\OperatorTok{/}\FloatTok{1e6}\NormalTok{, }\DecValTok{0}\NormalTok{), }\DataTypeTok{area =} \KeywordTok{round}\NormalTok{(}\KeywordTok{sum}\NormalTok{(area), }\DecValTok{0}\NormalTok{))}

\CommentTok{# Rename GRIDCODE column}
\KeywordTok{names}\NormalTok{(scenarios)[}\DecValTok{1}\NormalTok{] <-}\StringTok{ "scenario"}
\end{Highlighting}
\end{Shaded}

\hypertarget{results-table}{%
\subsubsection{Results: table}\label{results-table}}

Now we can turn this into a table.

\begin{Shaded}
\begin{Highlighting}[]
\KeywordTok{st_drop_geometry}\NormalTok{(scenarios) }\OperatorTok
\StringTok{  }\KeywordTok{kbl}\NormalTok{(}\DataTypeTok{col.names =} \KeywordTok{c}\NormalTok{(}\StringTok{"Sea-level rise (m)"}\NormalTok{, }\StringTok{"Parcel count"}\NormalTok{, }\StringTok{"Net loss ($m)"}\NormalTok{, }\StringTok{"Area flooded (ha)"}\NormalTok{)) }\OperatorTok
\StringTok{  }\KeywordTok{kable_material}\NormalTok{(}\KeywordTok{c}\NormalTok{(}\StringTok{"striped"}\NormalTok{, }\StringTok{"hover"}\NormalTok{))}
\end{Highlighting}
\end{Shaded}

\begin{table}[H]
\centering
\begin{tabular}[t]{r|r|r|r}
\hline
Sea-level rise (m) & Parcel count & Net loss (\$m) & Area flooded (ha)\\
\hline
1 & 60 & 51 & 43\\
\hline
2 & 89 & 88 & 108\\
\hline
3 & 227 & 195 & 189\\
\hline
4 & 620 & 566 & 301\\
\hline
5 & 1275 & 909 & 354\\
\hline
6 & 1863 & 1215 & 392\\
\hline
7 & 2196 & 1372 & 436\\
\hline
8 & 2614 & 1555 & 465\\
\hline
9 & 2958 & 1712 & 486\\
\hline
10 & 3287 & 1881 & 512\\
\hline
\end{tabular}
\end{table}

\hypertarget{results-map}{%
\subsubsection{Results: map}\label{results-map}}

In the map, the numbers 1 through 10 represent the number of meters of
sea-level rise associated with each scenario and the intensity of the
color represents the amount of property value lost.

\begin{Shaded}
\begin{Highlighting}[]
\KeywordTok{tmap_mode}\NormalTok{(}\StringTok{"plot"}\NormalTok{)}
\KeywordTok{tm_shape}\NormalTok{(scenarios) }\OperatorTok{+}
\StringTok{  }\KeywordTok{tm_polygons}\NormalTok{(}\StringTok{"net_value"}\NormalTok{, }\DataTypeTok{title =} \StringTok{"Net loss ($m)"}\NormalTok{) }\OperatorTok{+}
\StringTok{  }\KeywordTok{tm_facets}\NormalTok{(}\DataTypeTok{by =} \StringTok{"scenario"}\NormalTok{, }\DataTypeTok{nrow =} \DecValTok{5}\NormalTok{, }\DataTypeTok{ncol =} \DecValTok{2}\NormalTok{) }\OperatorTok{+}
\StringTok{  }\KeywordTok{tm_layout}\NormalTok{(}\DataTypeTok{main.title =} \StringTok{"Inundation scenarios for downtown Santa Barbara"}\NormalTok{,}
            \DataTypeTok{legend.position =} \KeywordTok{c}\NormalTok{(}\StringTok{"right"}\NormalTok{, }\StringTok{"bottom"}\NormalTok{),}
            \DataTypeTok{main.title.size =} \FloatTok{0.8}\NormalTok{)}
\end{Highlighting}
\end{Shaded}

\includegraphics{esm_263_hw2_pdf_files/figure-latex/map2-1.pdf}

\end{document}
